\documentclass{article}
\usepackage[a4paper, total={6in, 8in}]{geometry}

\usepackage{amsmath}
\usepackage{amssymb}
\usepackage{amsthm}
\usepackage{changepage}
\usepackage{gauss}
\usepackage{hanging}
%\usepackage{mathabx}
\usepackage{mathtools}
%\usepackage{siunitx}
\usepackage{systeme}
\usepackage{titlesec}

\usepackage{kotex}
\setmainhangulfont{NanumMyeongjo}

\renewcommand{\thesection}{\#\arabic{section}}
\renewcommand{\thesubsection}{(\arabic{subsection})}
\titleformat{\section}[runin]{\bfseries}{\thesection}{0.5em}{}
\titleformat{\subsection}[runin]{\normalfont\bfseries}{\thesubsection}{0.5em}{}


\newenvironment{claim}{\par\underline{Claim:}\qquad}{\par}
\newenvironment{claimproof}{\par\underline{Proof:}\qquad}{\par}
\newenvironment{solution}{\par\bigskip\noindent\textbf{Solution}\quad}{\par}
\newtheorem*{theorem*}{Theorem}
\theoremstyle{definition}
\newtheorem{proofpart}{Part}[section]
\renewcommand\theproofpart{\arabic{proofpart}}
\newtheorem{proofcase}{Case}[proofpart]
\renewcommand\theproofcase{\arabic{proofcase}}

\newcommand{\legendre}[2]{(\frac{#1}{#2})}
\newcommand\numberthis{\addtocounter{equation}{1}\tag{\theequation}}

\newcommand{\T}{\rule{0pt}{2.6ex}}       % Top strut
\newcommand{\B}{\rule[-1.2ex]{0pt}{0pt}} % Bottom strut

\newcommand{\then}{\Rightarrow\qquad}

\newcommand{\rrs}[1]{\mathbb{Z}/{{#1}\mathbb{Z}^{*}}}

\newcommand\lstrut{\rule{0pt}{2.5ex}}


\begin{document}
\noindent
\large\textbf{HW \#2} \\
\textbf{20120085 엄태경} \\
\normalsize CSE4170 기초 컴퓨터 그래픽스\\
법선 벡터와 평면에 대한 아핀 변환\\
2019. 4. 18.

\par\bigskip\bigskip
(전제). 기하 물체의 한 점 $p = (x, y, z, 1)^T$은 어떤 아핀 변환 행렬 $M_{4\times4}$에 의해 점 $p^\prime = Mp$로 변환된다.

%Q1
\section{}
\textbf{$p$에서의 법선벡터 $n = (n_x, n_y, n_z, 0)^T$는 $M$에 의해 $(M^{-1})^T n$으로 변환됨을 증명하라.}
\begin{solution}
\begin{sloppypar}
법선 평면 상의 임의의 점을 $p_0 = (x_0, y_0, z_0, 1)^T$이라 하자. 그러면 $p-p_0 = {(x-x_0, y-y_0, z-z_0, 0)^T}$는 법선 평면 상의 벡터이다. 즉 $p-p_0$는 $n$과 수직을 이룬다. 따라서,
\end{sloppypar}
\begin{equation}
	n^T (p-p_0) = 0 \label{eq:1}
\end{equation}
$p, p_0, n$에 변환 $M$을 적용한 결과를 각각 $p^\prime = (x^\prime, y^\prime, z^\prime, 1)^T, p_0^\prime=(x_0^\prime, y_0^\prime, z_0^\prime, 1)^T, n^\prime=(n_x^\prime, n_y^\prime, n_z^\prime, 0)^T$으로 두자. 그러면 전제에 의해,
\begin{align}
	p^\prime &= Mp \label{eq:2} \\
	p_0^\prime &= Mp_0 \label{eq:3}
\end{align}
$M$이 역변환이 존재하는 변환이라고 가정하면, $M$의 역행렬 $M^{-1}$이 존재한다. \eqref{eq:2}, \eqref{eq:3}의 양변에 $M^{-1}$을 곱하면,
\begin{align}
	M^{-1}p^\prime &= p \label{eq:4} \\
	M^{-1}p_0^\prime &= p_0 \label{eq:5}
\end{align}
\eqref{eq:4}, \eqref{eq:5}를 \eqref{eq:1}에 대입하면,
\begin{equation}
	n^T(M^{-1}p^\prime - M^{-1}p_0^\prime)=0 \label{eq:6}
\end{equation}
\eqref{eq:6}의 좌변은 다음과 같이 변형할 수 있다.
\begin{equation}
	\begin{aligned}
		n^T(M^{-1}p^\prime - M^{-1}p_0^\prime) &= n^T M^{-1} (p^\prime - p_0 ^\prime) \\
		&= ((M^{-1})^T n)^T (p^\prime - p_0 ^\prime)\\
		&= 0
	\end{aligned} \label{eq:7}
\end{equation}
\eqref{eq:7}에 의해 벡터 $(M^{-1})^T n$가 벡터 $p^\prime-p_0^\prime$와 수직을 이룬다. 이때 $p^\prime-p_0^\prime$은 변환된 법선 평면 상의 임의의 벡터이다. 따라서 변환된 법선 벡터는 다음 식으로 표현된다.
\begin{equation}
	n^\prime = (M^{-1})^T n \label{eq:8}
\end{equation}\qed
\end{solution}

%Q2
\section{}
\textbf{임의의 평면 $ax+by+cz+d=0$에 $M$을 적용하여 얻은 평면 $a^\prime x+b^\prime y+c^\prime z+d^\prime = 0$에 대해, $(a^\prime,b^\prime,c^\prime,d^\prime) = (a,b,c,d)M^{-1}$임을 증명하라.}
\begin{solution}
평면 $ax+by+cz+d=0$의 법선 벡터를 $n$이라 하면, $n = (a, b, c, 0)^T$이다. 이 평면 위의 한 점을 $p_0 = (x_0, y_0, z_0, 1)$이라 하면, 다음이 성립한다.
\begin{equation}
	ax_0 + by_0 + cz_0 + d = 0 \label{eq:9}
\end{equation}
\eqref{eq:9}를 다음과 같이 정리할 수 있다.
\begin{equation} \begin{aligned}
	d &= -(ax_0 + by_0 + cz_0)\\
	&= -(a,b,c,0)^T (x_0,y_0,z_0,1) \\
	&= -n^T p_0 \label{eq:10}
\end{aligned}\end{equation}
그러면 평면 $ax+by+cz+d=0$은 다음 식을 만족하는 점 $p=(x,y,z,1)$의 집합이라고 할 수 있다.
\begin{equation}
	n^T p -n^T p_0 = n^T (p-p_0) = 0 \label{eq:11}
\end{equation}
$p, p_0, n$에 변환 $M$을 적용한 결과를 각각 $p^\prime = (x^\prime, y^\prime, z^\prime, 1)^T, p_0^\prime=(x_0^\prime, y_0^\prime, z_0^\prime, 1)^T, n^\prime=(a^\prime, b^\prime, c^\prime, 0)^T$으로 두자. 그러면 위와 같은 방법으로,
\begin{equation}
	d^\prime = -n^{\prime T} p_0^\prime \label{eq:12}
\end{equation}
또, \eqref{eq:8}로 부터,
\begin{equation}
	n^{\prime T} = n^T M^{-1} \label{eq:13}
\end{equation}
\eqref{eq:12}에 \eqref{eq:3}과 \eqref{eq:13}을 대입하고 \eqref{eq:10}을 적용하면,
\begin{equation} \begin{aligned}
	d^\prime &= -n^{\prime T} p_0^\prime \\
	&= -n^T M^{-1} Mp_0 \\
	&= -n^T p_0 \\
	&= d
\end{aligned}\end{equation}
한편, $M$은 아핀 변환 행렬이므로 $M^{-1}$도 아핀 변환 행렬이고, 아핀 변환 행렬의 제4열은 $(0,0,0,1)^T$이다. 따라서,
\begin{equation}
	(0,0,0,d^\prime) = (0,0,0,d^\prime) M^{-1} = (0,0,0,d) M^{-1} \label{eq:15}
\end{equation}
\eqref{eq:13}, \eqref{eq:15}을 적용하면,
\begin{equation}\begin{aligned}
	(a^\prime, b^\prime, c^\prime, d^\prime) &= (a^\prime, b^\prime, c^\prime, 0) + (0, 0, 0, d^\prime)\\
	&= n^{\prime T} + (0,0,0,d) M^{-1}\\
	&= n^T M^{-1} + (0,0,0,d) M^{-1}\\
	&= (a,b,c,0) M^{-1} + (0,0,0,d) M^{-1} \\
	&= (a,b,c,d) M^{-1}
\end{aligned}\end{equation}\qed
\end{solution}
\end{document}